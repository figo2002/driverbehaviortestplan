
Driver is the most active part of road traffic system. Different drivers differ in driving behaviors, a driver also adapts his behavior to different external or internal states. For a long time in China, all drivers are professional drivers. However, with the boom of private cars, a lot of regular drivers emerged. Regular drivers show different driving behaviors in comparison with professional drivers. It is hopefully that through the investigation of the impact of driver behavior characteristics on traffic flow, a better understanding of driver behavior characteristics and the connection between driver behavior and traffic flow will be achieved.



A field car-following experiment featuring GPS and laser ranging was carried out to collect information about driver behaviors for both professional and regular drivers. Drivers recruited in the experiment were monitored for their speed, acceleration and following distance while they drove the experiment car on typical urban road in good weather conditions. After acquiring the trajectory and related data, aggregate analyses on the separation point of free driving and car-following , speed versus following distance,acceleration versus relative motion and model calibration were performed to investigate the heterogeneity of drivers' behaviors across the professional and regular driver groups. Finally, traffic simulation based on the SUMO software package was performed to asses the impact of 3 effects of driver behavior parameters' variations and combinations on traffic flow efficiency and safety measures.



Aggregate analyses suggest a difference in the separation point of free driving and car-following across the two driver groups. The results for model calibration suggest drivers have different desired speed and maximum deceleration rate. The simulation results indicate desired speed changes the shape of speed flow fundamental diagram in the free flow area. Higher desired speed increases the absolute value of time average inverse Time-to-Collision (abbreviated as |ITTC|/t), which means less safe. The increase of maximum deceleration rate makes it harder to reach the maximum capacity on density flow fundamental diagram. When the proportion of professional driver (with lower desired speed and higher maximum deceleration rate) increases, under same flow rate the maximum value of |ITTC| also increases, leading to a less safe state. However, there seem to be an optimal mixing rate of driver with different maximum deceleration rate where the |ITTC|/t is minimal.



In conclusion, both desired speed and maximum deceleration rate has impact on traffic flow efficiency and safety measures. And the main factor is maximum deceleration rate as its effects dominate the combined effects. Current evidences suggest there is a non-negligible effect of driver behavior characteristics on traffic flow.
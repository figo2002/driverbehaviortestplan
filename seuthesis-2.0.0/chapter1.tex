\chapter{绪论}
\section{立题的背景和意义}
%这是测试\cite{seugs:standard}

车辆的行驶是由驾驶人操纵有关机构实现的,驾驶人不仅是道路交通系统的信息处理者和决策者,也是道路交通系统的调节者和控制者,是道路交通系统中最活跃的要素。车辆的行驶状态是驾驶人所采取的驾驶行为的作用结果,而驾驶行为是在一定的外部环境下,驾驶人生理、心理和操作技能等驾驶特性相互作用的综合体现。不同驾驶人的生理状况和心理素质不同、驾驶操作技能也有差异,因此驾驶行为存在差异。不同驾驶人表现出不同的驾驶行为,同一名驾驶人在不同的外部环境和自身状态下的驾驶行为也不同。驾驶人行为特性具体的外在表现主要包括,在动态交通条件下,驾驶人的反应时间,速度选择,加速度选择,跟驰距离选择,变道决策与间隙接受等一系列特性。

长期以来,我国机动车为各种企事业单位所有,机动车驾驶人均为专业驾驶人,专业驾驶人的驾驶技能熟练程度、驾驶特性等方面的差异都较小。随着大量小汽车进入家庭,产生了大量的非专业驾驶人,非专业驾驶人在生理、心理以及驾驶操作技能等方面与专业驾驶人都存在明显的差异。而在传统的道路交通规划、设计与交通管理的研究和工程设计中,从宏观上假设驾驶人的驾驶特性相似和受同一普遍规律影响。这种假设没有体现驾驶人操作活动在复杂环境下的不确定性和不一致性,没有反映同一驾驶人在不同环境下驾驶行为特性的差异,没有反映大量驾驶人驾驶特性的微观差异对宏观交通流的影响,没有反映出驾驶人不同的驾驶特性对道路交通流的影响作用关系,这与现今真实的交通状况存在显著的差异。在此基础上进行的道路交通规划、设计与交通管理难以符合真实的交通现象,降低了交通规划、设计方案与交通管理措施的合理性与实施效果,同时也增加了交通事故的潜在危险性。

通过研究驾驶人行为特性对交通流的影响,将有助于对驾驶特性、驾驶行为与道路交通流的影响关系有更加深刻的理解;将有助于从人的因素的角度分析各种复杂交通流现象的形成和演化过程;将有助于确定影响道路交通安全的人的因素,为进一步探究改善交通流状况和提高交通安全性提供有力的理论支持;也可以为智能车辆、驾驶辅助支持系统等设备的研制提供理论支持。总之,研究驾驶人行为特性对道路交通流的影响,对提高道路交通系统运输效率和交通安全性具有重要意义,对我国今后的道路交通规划、设计、管理、控制也有重要的理论和应用价值。



\section{国内外研究概况}

\subsection{驾驶人行为特性研究}

Kou(1997)研究了高速公路汇入匝道处驾驶人的跟驰和汇入时的间隙接受行为。研究发现驾驶人匝道处的汇入行为,并不与任何单一的参数如汇入速度、主线流量、跟车间距与时距、速度差等显著相关,而各参数的组合能更好的解释驾驶人的驾驶行为\cite{Kou1997}。Salvucci和Liu(2002)对驾驶人变道过程中的操纵和眼动情况进行了分析\cite{Salvucci2002}。Hamdar(2004)在Gipp’s跟驰模型的基础上融入了紧急条件下驾驶人的不同驾驶类型,并对驾驶人的特性进行了敏感性分析\cite{Hamdar2004}。Yang(2007)通过城市道路、乡村道路和高速公路的实测数据对自由流和非自由条件下的驾驶人的速度选择和加减速行为进行了研究,并对不同行为的驾驶人进行了分类。研究发现对于不同的道路驾驶人的速度行为差异明显,并与交通流条件高度相关;驾驶人一般会选择高于限速值不超过10\%的速度;根据驾驶人群体可分为进攻型、防守型、和一般型三类,进攻型倾向于选择高于限速的车速,防守型相反,一般型选择不超过限速太多的车速;驾驶人的加减速选择在非自由流条件下变化更大;在自由流条件下,驾驶人的加减速依赖于瞬时速度。在中低速情况下,加减速率随速度增加而减小。而在高速条件下较为稳定;不同类型驾驶人的加减速选择差异明显,进攻型驾驶人加减速率较大而防守型相反\cite{Yang2008}。Xu(2007)研究了环形交叉口驾驶人的间隙接受行为,研究从视频数据中提取事件信息,用最大似然法和Raff法对临界车头时距和跟车时距进行了估计并进行了比较。研究发现交织流量和环道车速与临界车头时距和跟车时距呈现负相关性\cite{Xu2007}。Heaslip(2007)在跟驰模型中加入了驾驶人对车道改变的熟悉程度、适应性、攻击性,研究了施工区域驾驶人的驾驶行为\cite{Heaslip2007}。Dalia Said(2008)根据高速公路不同路段的实测数据,从驾驶人劳动强度的角度分析了驾驶人的加减速和方向盘操作行为,并与高速公路线形设计的安全性进行了关联性分析\cite{Said2008}。Yeo(2008)根据实际轨迹数据研究了驾驶人的微观非对称行为,在驾驶模型中加入了操作差错和驾驶人预期,解释了交通流的滞后、不稳定、通行能力下降、换道松弛和时走时停现象,并分析了基本图中的D型和A型曲线\cite{Yeo2008}。

孔繁森 (2004) 依据Kuipers的定性仿真方法对熟练驾驶人和刚学会开车的驾驶人,在具有不同曲率半径和路面情况的一段道路上所做出的车速选择做了定性推理,给出了两类不同驾驶人驾车行为的定性状态描绘图和相应的车速曲线变化图\cite{孔繁森2004}。张开冉 (2008) 对新驾驶人的反应时进行测试,与一定样本量的对照组相应指标作比较,结果表明信息复杂度较大时,两组间反应时差异较为显著\cite{张开冉2008}。王晶 (2008) 基于驾驶人性别对城市快速路跟驰模型进行了研究\cite{王晶2008}。范红静 (2008) 研究了真实道路交通环境下专业与非专业驾驶人动态视觉特性的差异,并提出了满足驾驶人行车安全的交叉口通行能力的计算方法。研究发现:驾驶经验与交通流状态都是影响驾驶人动态视觉特性的主要因素;城市道路自由流和拥挤流环境下以及城市道路、近郊公路、高速公路三种类型道路上专业与非专业驾驶人的动态视觉特性存在明显差异\cite{范红静2008}。武睿(2008)研究了不同类型驾驶人在不同交通环境下观察不同交通标志的眼部运动特征参数,研究发现:男性驾驶人对交通标志的视认性明显优于女性驾驶人;驾驶人的年龄和驾龄对交通标志的视认存在明显影响\cite{武睿2008}。徐上(2009)在实际道路交通环境下进行了驾驶人跟驰行为实验,测试了驾驶人的跟驰距离、跟驰时距等操作特征参数。研究发现:性别因素、驾驶人精神状态因素、驾龄和累计行驶里程对跟驰时距有较为显著的影响;男性驾驶人跟驰时距较女性驾驶人平均低约 0.7s;驾龄和累计行驶里程越长,跟驰时距相应越小,但并非是线性减少关系\cite{徐上2009}。李娅(2009)研究了不同光照和行车速度条件下专业与非专业驾驶人对指路标志视认的眼部运动特征参数和标志视认距离。研究发现:行车速度、光照条件和驾驶经验对指路标志的视认距离存在较为明显的影响,光照条件和驾驶经验的影响尤为显著\cite{李娅2009}。裴玉龙 (2009) 以哈尔滨市小客车和大客车驾驶人作为观察对象拍摄记录了他们在车道变换进程中的视点变换情况。分析了驾驶人各视点停留概率和两视点之间转移概率的分布规律,得到驾驶人在车道变换进程中的注意力分配特性\cite{裴玉龙2009}。

\subsection{驾驶人与驾驶行为建模}

Ahmed(1999)对驾驶人的加减速和变道行为进行了建模\cite{Ahmed1999}。Oliver和Pentland(2000)在测试车数据基础上利用组对隐马尔科夫模型对驾驶人行为进行了识别和预测\cite{Oliver2000}。Chen和Ulsoy(2001)利用驾驶模拟器的测试数据对驾驶人的转向操作和其中的不确定型进行了建模和分析\cite{Chen2001}。Y. Lin等(2005)用神经网络对车辆-驾驶人-环境系统进行建模,并进行了模拟\cite{Lin2005}。Dario D.Salvucci(2006)基于ACT-R的认知模型对驾驶人的转向操作、跟驰、变道准备和变道行为、注视分布进行了研究\cite{Salvucci2006}。Tomer Toledo 等(2007)提出了一个整合加速、车道变换、间隙接受等各种驾驶行为模型的框架\cite{Toledo2007}。Samer Hani Hamdar(2009)对驾驶人行为进行建模,把驾驶行为作为一种随机的冒险行为进行研究,从而将人的认知过程纳入到微观交通流模型中,并且发现驾驶人预期、驾驶人行为差异和驾驶经验长短对其行为有显著影响\cite{Hamdar2009a}。

王家波 (2001) 提出了一种符合人的特性的驾驶人PID模型,并进行了仿真计算\cite{王家波2001}。李谦 (2003) 对最优预瞄加速度决策模型存在的问题加以修正,建立了一个可以用于在高速行驶工况下复杂车辆模型横向与纵向综合控制的驾驶人控制行为模型\cite{李谦2003}。邱凌云 (2005) 基于Agent理论对驾驶人进行建模.研究了Agent的跟驰、换道和挤占道等行为\cite{邱凌云2005}。杨新月 (2006) 基于认知活动链对驾驶人行为进行了建模和仿真\cite{杨新月2006}。朱山江 (2006) 采用北京实测数据,标定了改进的智能驾驶人模型,并应用这个模型研究了快速路出口车队遇到下游信号灯路口后排队和拥堵的产生机制\cite{朱山江2006}。吴超仲 (2007) 以概率反映驾驶人性格特性分布,用权重系数表示不同性格驾驶人在操作方面的差异, 建立了反映驾驶人性格的车辆跟驰模型\cite{吴超仲2007}。张磊 (2008) 基于神经网络方法建立了一种集成式驾驶人跟车模型\cite{张磊2008}。许骏 (2008) 将驾驶人-汽车看作统一的人机系统,建立了基于Markov决策过程的驾驶人行为模型,并对所建模型进行了计算机仿真\cite{许骏2008}。王晓原 (2008) 基于决策树对驾驶行为决策机制进行了研究\cite{王晓原2008}。刘兵 (2008) 基于驾驶人视知觉对驾驶人的车速控制和车道保持机理进行了研究\cite{刘兵2008}。李勇 (2008) 综合了驾驶人反应时间、驾驶技能、性格特性和车辆特性等因素,考虑了车辆在不同路段的不同行驶特征建立了微观的驾驶人模型\cite{李勇2008}。


\subsection{驾驶人行为与交通流关系研究}

Bonzani(2000)研究了一阶流体力学模型中Diffusion项与驾驶人行为的关系,提出了改进驾驶人行为模型的方法\cite{Bonzani2000}。Hoogendoorn, S. P和 P. H. L. Bovy(2000)在Daganzo研究的基础上提出了多类用户的气动理论连续交通流模型,将速度方差加入到偏微分方程中,研究了各类用户相互竞争的行为\cite{Hoogendoorn2000}。Bellomo, N., A. Marasco(2002)回顾了一阶与二阶的流体力学交通流模型,讨论了二阶模型中局部平衡车速对驾驶人刺激的参数识别问题,提出了流体力学模型中关于驾驶人的可研究的方向,提出了驾驶人环境作用力的建模方向\cite{Bellomo2002}。Zhang, H. M. (2003)研究了微观跟驰模型与宏观流体力学模型的联系,研究发现驾驶人记忆导致宏观模型中的粘性现象,并基于此建立了带有粘性的二阶流体力学模型,通过各向异性因子统一了LWR和PW等其他流体力学模型\cite{Zhang2003a}。H. B. ZHU等(2007)建立了一个包括不同驾驶人概率的元胞自动机模型,分析了不同驾驶人混合对交通流的影响\cite{H.B.ZHU2007}。H.B. Zhu 和 S.Q. Dai(2008)研究发现当驾驶人检测车头距的反应延误增加时,交通流稳定区域减小,驾驶人检测车头距的反应延误在交通拥堵转换中扮演重要的角色\cite{Zhu2008}。

雷丽等(2003)在一维交通流元胞自动机NaSch模型的基础上,优先考虑驾驶员的不确定性敏感预期行为,将随机延迟过程放在确定性减速之前,从而建立一种新的一维元胞自动机交通流模型,研究发现敏感驾驶因素对车流的作用很大,随着敏感驾驶车辆的增多,道路容量随之提高\cite{雷丽2003}。葛红霞等(2005)研究发现当考虑快车和慢车的混合交通流时,即使少量的慢车也会导致交通流量大幅度下降\cite{葛红霞2005}。李启朗 (2006) 在单车道元胞自动机交通流NS模型基础上,通过引入不同的刹车概率来反映不同驾驶人的驾驶特性,并在周期边界条件下,对由激进驾驶车辆和谨慎驾驶车辆构成的混合交通流进行模拟.结果表明,在有谨慎驾驶车辆构成的交通流的临界密度以前,混合交通流的流量完全由谨慎驾驶人的特性决定\cite{李启朗2006}。祝会兵(2008)研究发现:当道路上有一些比较谨慎的司机时,随机减速概率就大,道路流量明显下降;没有事故车瓶颈和其它瓶颈,慢车对交通流特性有很大的影响;司机的敏感性越差,反应延迟时间越长,越容易产生交通阻塞\cite{祝会兵2008}。王欣等(2008)建立一种以车头间距和驾驶人反应时间等为参数的回波速度和速度-密度关系模型,研究发现驾驶人反应时间变化是产生速度陡降现象的根本原因\cite{王欣2008}。


\subsection{国内外研究总结}

国内外针对驾驶人行为的研究集中在驾驶人的认知行为和操作行为上,对认知行为的研究主要包括驾驶人眼动特性,速度、距离等感知特性,驾驶人注意分配特性,驾驶人生理数据的采集分析等。对驾驶人操作行为的研究主要包括驾驶人的一般与特定道路条件下的跟驰行为,换道行为、超车行为等。在现有的研究中,对驾驶人行为差异性的研究基本都是在静态条件下进行比较,缺乏对实际道路交通环境下,车辆动态行驶过程中驾驶人决策和行为的差异性的研究。

国内外针对驾驶人与驾驶行为的建模,逐渐从机械的力学模型发展到控制理论模型,再到跟驰模型和变道模型以及两者相结合的综合模型,在交通流模型中越来越多地考虑到了驾驶人的主观因素,现今已有一些研究考虑了驾驶人的认知和决策过程,有的模型中考虑了驾驶人根据对周围车辆动态的预计和判断对车辆进行操控这一特性,使得驾驶人模型更为符合实际。

国内外对驾驶人与交通流关系的研究主要从两个方面开展,一方面,从微观的角度,通过实际数据的测量以及使用基于驾驶人特性的微观模型进行仿真,通过调整驾驶人行为特性参数来研究驾驶人特性对交通流的影响。另一方面,从宏观交通流模型修正的角度,通过对宏观模型中驾驶人因素项的改进来解释交通流中的幽灵拥堵、相位变化等交通流现象。%两方面的研究均取得了一些成果,但是微观的驾驶人模型与宏观交通流模型之间似乎缺少某种联系,有关研究表明,这种联系可能会存在于驾驶人之间的相互作用之中。

\section{研究内容和论文框架}

\subsection{论文依托}

本论文依托于国家自然科学基金项目:“驾驶人员驾驶特性对道路交通流的影响机理”(50708019),以及霍英东教育基金优选项目:驾驶人员驾驶特性研究(104010)开展研究。

\subsection{研究内容}
(1)驾驶人驾驶行为特性主要参数和交通流参数采集

以专业和非专业驾驶人为研究对象,调查其性别、年龄、驾龄、行驶里程等基本信息。对其进行相关的心理学实验以研究其决策行为。选择典型的城市道路单向两车道和单向多车道路段,在良好天气条件下,采集实际交通流状态下驾驶人的跟车速度、跟车间距、加速度、变道间隙接受等驾驶行为的主要参数和相应道路交通流的平均速度、平均密度、平均车速差等交通流参数。

(2)驾驶人行为特性分析

根据实验所得的测试车轨迹数据与其邻近车辆的相对速度和相对位置的时间序列数据,分析驾驶人在其动态变化的驾驶环境中的反应时间,速度选择,加速度选择,跟驰距离选择,变道决策与间隙接受等一系列特性。比较驾驶动态过程中的不同驾驶人行为的差异性,根据驾驶人行为的差异性对驾驶人进行分类,并与驾驶人的心理实验数据进行关联性的分析。

(3)驾驶人行为特性对交通流影响分析

根据实验所得数据,对驾驶人驾驶行为进行建模。建模时考虑驾驶人对其他车辆运动的预计性和驾驶人周围车辆的动态对驾驶人驾驶行为的影响,考虑相邻驾驶人的相互作用。对于不同行为特性的驾驶人采用不同的结构或不同的参数进行模型的建立。

基于所建立的驾驶人的驾驶行为模型,编写仿真程序,针对不同情况分别进行模拟研究:1)研究不同类型驾驶人的行为对其周围局部交通流的平均车速、平均密度、平均车速差等参数的影响。2)通过对驾驶人模型参数以及不同类型驾驶人混合比例的调整,通过模拟,研究驾驶人行为特性微观差异对于路段交通流宏观特性的影响。




\subsection{论文框架}

%\begin{table}[htbp]
% \centering
% \caption{Add caption}
% \begin{tabular}{cccc}
%   \addlinespace
%    \toprule
%    Face database & Yale  & Caltech & ORL \\
%    \midrule
%    Number of training & \multicolumn{ 1}{c}{5} & \multicolumn{ 1}{c}{3} & \multicolumn{ 1}{c}{3} \\
%    samples per class  & \multicolumn{ 1}{c}{} & \multicolumn{ 1}{c}{} & \multicolumn{ 1}{c}{} \\
%    \bottomrule
%    \end{tabular}
%  \label{tab:addlabel}
%\end{table}

驾驶人是道路交通系统中最活跃的要素。不同驾驶人驾驶行为存在差异,同一名驾驶人在不同的外部环境和自身状态下的驾驶行为也不同。长期以来,我国机动车驾驶人均为专业驾驶人,随着大量小汽车进入家庭,产生了大量的非专业驾驶人。非专业驾驶人驾驶操作行为方面与专业驾驶人都存在差异。通过研究驾驶人行为特性及其对交通流的影响,将有助于对驾驶特性、驾驶行为与道路交通流的影响关系有更加深刻的理解。

本文分别以专业和非专业驾驶人为研究对象,选择典型的城市道路,在良好天气条件下,通过使用GPS和激光测距仪的跟车实验采集实际交通流状态下驾驶人的跟车速度、跟车间距、加速度等驾驶行为的主要参数。根据实验所得的测试车轨迹数据与其邻近车辆的相对速度和相对位置的时间序列数据,分析了驾驶人在真实驾驶环境中的自由行驶与跟驰的临界点特性、跟驰距离选择、加速度选择特性。通过对数据的集计分析以及跟驰模型的参数标定,分析并比较了驾驶过程中的不同驾驶人行为的异同。最后应用交通仿真程序SUMO,针对不同情况分别进行模拟研究。通过对驾驶人模型参数以及不同类型驾驶人混合比例的调整,研究了驾驶人行为特性对路段交通流的效率和安全特性的影响。

研究结果表明专业和非专业两组驾驶人的跟驰与自由流临界点特性方面存在差异。参数标定结果表明,驾驶人在期望速度和最大减速度方面存在差异。模拟研究发现,期望车速的改变对速度流量基本图上的曲线形状产生影响,且这种影响主要在自由流阶段。期望速度的增加导致碰撞时间TTC倒数的绝对值的时间平均值增加,这表明期望速度的增加对交通流的安全性存在负面影响。最大减速度的增加导致密度流量图上较难达到通行能力的区域。随着最大减速度的增加TTC倒数绝对值呈现先减小后增大的趋势,这表明存在不同最大减速驾驶人的最佳混合比例,从而使得TTC倒数绝对值的时间密度最小。专业驾驶人(低期望速度,高最大减速度)的增加导致相同流量下TTC倒数绝对值的最大值增加,交通流偏向不安全的方向。

本文的结果表明期望车速及最大减速度对交通流的效率性及安全性存在不同特点和程度的影响,而主要的影响因素是驾驶人的最大减速度。现有证据表明驾驶人行为特性对交通流的效率性和安全性存在不可忽略的影响。
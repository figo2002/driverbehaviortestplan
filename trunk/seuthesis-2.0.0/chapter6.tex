\chapter{结论与展望}

本文在假设专业与非专业驾驶人驾驶行为存在差异的基础上,通过自行设计的实验,收集了大量的数据,分析了专业驾驶人与非专业驾驶人驾驶行为特性的异同,并通过微观模拟仿真的方法研究了驾驶人行为特性对交通流的可能影响。

\section{主要成果和结论}
\subsection{研究方法}
设计并应用车载激光测距仪、车载GPS卫星定位系统、摄像机同步实时监测实验车辆的驾驶人行为特性关键指标,获取大量真实有效的实验数据。获得了跟车实验的大概的有效数据获取率为40\%,为进一步研究提供了数据量的参考。

给出了使用跟驰模型参数标定和Bootstrap方法估计驾驶人参数的方法框架,以及通过模拟分析驾驶行为特性对交通流影响的思路。

\subsection{关于驾驶人行为特性的发现}
研究表明,驾驶人所采取的减速度与TTC倒数呈现较为明显的线性关系,驾驶人在加减速方面存在不对称性。

专业和非专业两组驾驶人的跟驰与自由流临界点特性方面存在差异。通过参数标定表明,驾驶人在期望速度和最大减速度方面存在差异。


\subsection{驾驶人行为特性对交通流的影响}

最小通行能力似乎取决于所有驾驶人中的最低期望车速。期望车速对基本图上的速度流量曲线的形状产生影响,且主要在自由流阶段。

高期望速度导致TTC倒数绝对值的时间密度增加,低期望车速驾驶人混入少于高期望车速驾驶人时,TTC倒数绝对值的时间密度变化较大。


最大减速度的增加,导致密度流量图上较难达到通行能力的区域,且在J线以上较为不稳定的同步流状态。

最大减速的增加导致总模拟时间有增加的趋势。少量高最大减速度的驾驶人的混入时,即使模拟时间较大增加,并且模拟时间的变化范围较大。

对于具有不同最大减速度的驾驶人,似乎存在最佳的混合比例使得TTC倒数绝对值的时间密度最小。

混合作用影响下随着B型驾驶人(代表专业驾驶人)的增加,TTC倒数的绝对值的最大值在相同流量下有增加的趋势,表明相同流量下专业驾驶人的增加导致交通流偏向不安全的方向。

期望速度与最大减速度共同影响时,最大减速度影响的效应占了主要成分,表明相比期望速度最大减速度是影响交通流关键的因素。


\section{研究所存在的问题}

由于本文收集的驾驶人数据量,就驾驶人个数而言还较小,这导致样本的代表性存在不足,例如样本中女性驾驶人较少。

样本中专业组和非专业组驾驶人除了在驾驶经验方面存在差异之外在平均年龄上也存在差异,因此驾驶行为的差异可能不是由于驾驶经验的差异导致的。

本文中由于实验条件的限制未对变道行为进行更多讨论。

%由于不存在完全模拟现实的模型存在,模拟所得出的结果也没有完备的理论支撑,因此结果应当审慎地对待。



\section{研究展望}

可以考虑改进研究设备,例如增加激光测距仪的光线条数来获取变道行为的数据。要获得更确定性的结论,应当对更多涉及范围更广的驾驶人进行更长时间的研究。

可以考虑使用其他模型或者通过实际测试的方法来验证驾驶人行为特性对交通流的影响。

研究表明驾驶人的减速度与交通流关系密切,这为辅助驾驶系统的研究提供了更多的动机和条件。

%更多的考虑交通流的安全性的评价

%(1)	设计了一种新的研究驾驶人行为特性的实验方法
%在对驾驶人行为特性定义与分类的基础上,应用车载激光测距仪、车载GPS卫星定位系统、摄像机同步实时监测实验车辆的驾驶人行为特性关键指标,利用调查问卷了解必要的驾驶人心智特性。本实验能够克服传统调查实验中的各种不足与缺点,能够获取大量真实有效的实验数据,在国内是首次采用车载激光测距仪、GPS和摄像机相结合的方式同步实时监测实验车辆。
%(2)	研究了城市道路上驾驶人行为特性的规律
%针对驾驶人行为特性的关键指标,研究了在单向两车道与单向三车道城市道路上,驾驶人在速度、车头间距和车头时距方面的规律。分析了速度、车头间距和车头时距的分布及其统计特征值,通过对车头间距和车头时距统计值的进行回归分析,得到专业驾驶人和非专业驾驶人的车头间距与速度之间关系的拟合函数和车头时距与速度之间关系的拟合函数。
%(3)	建立的驾驶人行为特性对城市路段通行能力的修正模型
%利用专业驾驶人与非专业驾驶人车头间距与速度之间关系的拟合函数和车头时距与速度之间关系的拟合函数,分别得到基于车头间距与车头时距的路段通行能力模型,并给出了两种路段通行能力模型在各速度下,混合交通流的通行能力驾驶人修正系数推荐值。为了消除误差,取两者的平均值作为最后推荐值,得到基于驾驶人行为特性的城市路段通行能力模型,并给出不同道路条件各速度下,驾驶人混合交通流的通行能力驾驶人修正系数推荐值。
%(4)	对论文提出的通行能力模型与通行能力修正系数进行仿真
%通过仿真程序对论文提出的城市道路路段通行能力模型和推荐的通行能力修正系数进行模拟检验,可视、直观的表现出驾驶人混合交通流对城市道路路段通行能力的影响。

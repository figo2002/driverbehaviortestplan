历经一年的时间,论文已接近尾声,同时两年半的硕士研究生生活在这个季节也即将划上句号,而于我的人生却只是一个逗号,我将面对又一次征程的开始。在此,我要向我的家人、老师、同学、朋友表达我最诚挚的感谢!是你们让我感到了温暖,是你们让我有了前进的力量,是你们让我的读研生活如此精彩。
首先我要感谢我的指导老师陆建教授。在两年半的研究生生活中,陆老师无论在学术上,还是在生活中,都对我关怀备至,使我得到很大的提高。陆老师治学严谨,知识渊博,恭谦近人,是我的良师益友,更是我学习的楷模。在陆老师身上学习到的不仅是专业的知识,更多的是做人做事的道理,对我今后的人生道路将产生长远的影响。论文的完成得到陆老师无微不至的关心与指导,从论文的开题、实验的设计、仪器的准备到具体的研究工作,无不倾注了陆老师大量的心血!在论文完成之际,谨向尊敬的陆老师致以我最诚挚的谢意与衷心的祝福!
论文的完成还得到了姜军师兄和吴璠同学的大力支持,姜军师兄如兄长般的关怀,在关键时刻对我的指点,吴璠同学对我不遗余力的帮助,使得我能够顺利的完成论文。借此机会,在此表示我深深的谢意。
感谢王炜院长、过秀成老师、陈学武老师、任刚老师、陈峻老师、邓卫老师、李文权老师、陈琳老师、项乔君老师、胡晓健老师、王昊老师、杨敏老师,在东南大学的求学岁月里,教给了我学术上的相关专业知识以及探求知识的方法。
感谢秦霞书记、张建老师、陈怡老师、沈溶老师、黄宏老师、罗磊老师,为交通学院营造出一个大家庭的氛围,以及对我的关怀,让我在近七年的大学生活中,感受到切实的家一般的温暖。
感谢课题组的孙祥龙师兄、杨海飞师兄、赵勇同学、周成彦同学、李文华同学、杜璇同学对我的支持与照顾,感谢宿舍兄弟侯现耀、张磊、方程炜对我的帮助与关怀,能与你们共度这最美好的读研时光,是我的幸运与幸福。
最后特别要感谢我的父亲、母亲和姐姐,是你们一直以来默默无闻的付出和一贯的支持与鼓励,才使我能够一步一步的成长,才使我有信心和毅力完成学业,无论今后我身在何方,心中不变的思念,永远是相同的地方。
谨以此文献给所有关心、支持我的人,我将更加努力,不辜负大家的期望。
胡武林 
